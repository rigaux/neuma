
\begin{table*}[h]
  \centering
  \caption{Summary of the dataset content}
  \label{tab-dataset}

\begin{small}
\begin{tabular}{l|l|l|l|r|r|r|c|c|c|c|c}
\textbf{Title} & \textbf{Genre} 
    & \textbf{Images} 
	& \rotatebox[origin=c]{90}{\textbf{\#parts}}
    & \rotatebox[origin=c]{90}{\textbf{\#pages}}
	& \rotatebox[origin=c]{90}{\textbf{\#systems}}
    & \rotatebox[origin=c]{90}{\textbf{\#measures}}
    & \rotatebox[origin=c]{90}{\textbf{Lyrics}} 
	& \rotatebox[origin=c]{90}{\textbf{Clef changes}} 
	& \rotatebox[origin=c]{90}{\textbf{Ksign changes}} 
	& \rotatebox[origin=c]{90}{\textbf{Tsign changes}} 
	& \rotatebox[origin=c]{90}{\textbf{Cross-staff voices}} \\
\hline

{{line.title|safe}} & 
   {{line.genre}} &
\href{ {{line.iiif_link|safe}} }{Link} &
  {{line.nb_parts}} &
 {{line.nb_music_pages}} & 
  {{line.nb_systems}} &
  {{line.nb_measures}} & 
  {{line.with_lyrics}} &
  {{line.clef_changes}} &
  {{line.ksign_changes}} &
  {{line.tsign_changes}} &
  {{line.cross_staff_voices}} \\

\hline
Summary &  & & & {{total_pages}} & {{total_systems}} & {{total_measures}} \\
\hline
\end{tabular}
\end{small}
\end{table*}